\documentclass{article}
\usepackage[utf8]{inputenc}
\usepackage{amsmath}
\usepackage{hyperref}

\title{Resumen de Clase}
\author{Carlos}
\date{\today}

\begin{document}

\maketitle

\section{Introducción}
En esta clase, discutimos el componente llamado \textbf{Automatic Database Diagnostic Monitor (ADDM)}, que se encarga de diagnosticar de manera automática y proactiva el desempeño de la base de datos, identificando problemas y proponiendo soluciones.

\section{Funciones del ADDM}
\begin{itemize}
    \item Identificar áreas de la base de datos con grandes tiempos de procesamiento.
    \item Sugerir cambios a nivel de hardware (ej. más memoria) y a nivel de parámetros de la base de datos.
    \item Realizar configuraciones a nivel de esquemas y objetos creados por los usuarios.
    \item Reportar el resultado de los cambios aplicados para evaluar su efectividad.
\end{itemize}

\section{Vistas Relevantes}
La recolección y almacenamiento de datos se realiza en las siguientes vistas:
\begin{itemize}
    \item \texttt{DBA\_ACTIVE\_SESSION\_HISTORY}
    \item \texttt{V\$SESSION}
    \item \texttt{V\$ACTIVE\_SESSION\_HISTORY}
\end{itemize}

\section{Proceso BMNL}
El \textbf{BMNL} es un proceso auxiliar encargado de la sincronización de datos estadísticos desde la memoria hacia el disco. Este proceso almacena datos estadísticos a nivel de usuario y sesión.

\section{Ejercicios}
Se propusieron ejercicios para familiarizarse con las vistas mencionadas:
\begin{enumerate}
    \item Generar una sentencia SQL que muestre información de la sesión (usuario, máquina, estado, etc.).
    \item Mostrar el identificador y el texto SQL de la última sentencia ejecutada en la sesión.
    \item Generar una sentencia que muestre la fecha y el código SQL de todas las sentencias ejecutadas en la sesión a lo largo del tiempo.
\end{enumerate}

\section{Proceso RECOVER}
El \textbf{RECOVER Process} es un proceso de background utilizado en bases de datos distribuidas. Se encarga de:
\begin{itemize}
    \item Resolver transacciones que no pudieron completarse o que hicieron \textit{rollback}.
    \item Limpiar recursos y comunicar acciones a otros nodos involucrados en una transacción distribuida.
\end{itemize}

Ejemplo de transacción distribuida:
\begin{verbatim}
UPDATE local_table SET ...;
UPDATE remote_table@remote_db SET ...;
COMMIT;
\end{verbatim}

Este proceso utiliza el protocolo \textbf{Two-Phase Commit} para asegurar la correcta ejecución de transacciones distribuidas.

\section{Tareas y Próximos Pasos}
Para la próxima clase, se revisarán los ejercicios pendientes de los temas 4 y 5. Es importante recordar las fechas de entrega establecidas en la herramienta de \textit{Classroom}.

\end{document}

