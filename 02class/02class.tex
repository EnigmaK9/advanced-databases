\documentclass{article}
\usepackage[utf8]{inputenc}
\usepackage{amsmath}
\usepackage{hyperref}

\title{Resumen de Clase}
\author{Carlos}
\date{\today}

\begin{document}

\maketitle

\section{Introducción}
En esta clase se abordaron los siguientes puntos clave:

\begin{itemize}
    \item Saludo y revisión del estado actual de los alumnos.
    \item Discusión sobre el manejo de horarios mixtos en el nuevo semestre.
    \item Revisión del ejercicio práctico número 1 y resolución de dudas.
\end{itemize}

\section{Temas Principales}

\subsection{Herramientas de Administración}
\begin{itemize}
    \item Uso de herramientas gráficas y de línea de comandos para la administración de bases de datos.
    \item Programación en PL/SQL y Shell para automatizar tareas administrativas.
\end{itemize}

\subsection{Ejercicio Práctico Número 2}
\begin{itemize}
    \item Generación de un script en Shell para descargar imágenes de una lista de URLs y comprimirlas en un archivo zip.
    \item Validación de parámetros de entrada y manejo de errores.
    \item Ejemplo detallado de un script en Shell, explicando cada paso y su funcionalidad.
\end{itemize}

\section{Detalles del Ejercicio Práctico Número 2}

\subsection{Encabezado del Script}
Cada script debe incluir un encabezado con el autor, fecha y una breve descripción del script. Por ejemplo:
\begin{verbatim}
#!/bin/bash
# Autor: Carlos
# Fecha: 2024-07-01
# Descripción: Script para descargar imágenes y comprimirlas en un archivo zip.
\end{verbatim}

\subsection{Parámetros de Entrada}
El script recibe tres parámetros:
\begin{itemize}
    \item Ruta del archivo de texto con las URLs de las imágenes.
    \item Número de imágenes a descargar.
    \item Nombre del archivo zip a generar (opcional).
\end{itemize}

\subsection{Validaciones y Funciones}
\begin{itemize}
    \item Validación de la existencia y formato correcto del archivo de imágenes.
    \item Validación del rango del número de imágenes (1 a 90).
    \item Creación de la función \texttt{ayuda} para mostrar mensajes de error y terminar la ejecución en caso de parámetros incorrectos.
\end{itemize}

\subsection{Descarga de Imágenes y Compresión}
\begin{itemize}
    \item Uso del comando \texttt{wget} para descargar las imágenes.
    \item Manejo de errores durante la descarga.
    \item Creación del archivo zip y configuración de permisos.
    \item Generación de un archivo de texto con la lista de archivos incluidos en el zip.
\end{itemize}

\section{Conclusiones}
\begin{itemize}
    \item La programación en Shell es poderosa y permite automatizar tareas administrativas complejas.
    \item Es crucial validar y manejar correctamente los parámetros de entrada para asegurar la robustez del script.
    \item Los alumnos deben familiarizarse con los comandos y estructuras básicas de Shell para el curso.
\end{itemize}

\end{document}

