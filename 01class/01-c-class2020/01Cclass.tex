\documentclass{article}
\usepackage[utf8]{inputenc}
\usepackage{amsmath}
\usepackage{hyperref}

\title{Resumen de Clase}
\author{Carlos}
\date{\today}

\begin{document}

\maketitle

\section{Introducción}
En esta clase, discutimos el protocolo \textbf{Two-Phase Commit} y su importancia en la sincronización y organización de transacciones distribuidas en bases de datos.

\section{Protocolo Two-Phase Commit}
\begin{itemize}
    \item El proceso de una transacción se realiza en dos etapas:
    \begin{enumerate}
        \item \textbf{Primera fase: Preparación}
        \begin{itemize}
            \item Cada nodo de la base de datos realiza su tarea y se prepara para hacer \textit{commit}.
            \item Los nodos informan que están listos para realizar \textit{commit}.
        \end{itemize}
        \item \textbf{Segunda fase: Confirmación}
        \begin{itemize}
            \item Si todos los nodos reportan que su \textit{commit} fue exitoso, la transacción se considera exitosa.
            \item Cada nodo sincroniza los cambios a través del \textit{commit}.
        \end{itemize}
    \end{enumerate}
    \item Si algún nodo falla en realizar \textit{commit}, se ejecuta una operación de \textit{rollback} en todos los nodos.
\end{itemize}

\section{Proceso RECO}
\begin{itemize}
    \item El proceso \textbf{RECO (Recover Process)} se encarga de las operaciones de recuperación y limpieza en transacciones distribuidas.
    \item Si un nodo falla, \textbf{RECO} comunica a los otros nodos para realizar \textit{rollback}.
\end{itemize}

\section{Finalización del Tema 5}
Hemos concluido el estudio de los procesos de la base de datos, destacando los procesos principales y obligatorios para su funcionamiento.

\section{Otros Procesos}
\begin{itemize}
    \item Existen otros procesos opcionales que realizan diversas tareas adicionales.
    \item La vista \texttt{V\$PROCESS} contiene información detallada sobre todos los procesos en una instancia.
    \item Ejemplo de consulta en \texttt{V\$PROCESS}:
    \begin{verbatim}
    SELECT spid, pname, username, program
    FROM V$PROCESS;
    \end{verbatim}
\end{itemize}

\section{Examen y Ejercicios}
\begin{itemize}
    \item Próxima semana: revisión de ejercicios de los temas 4 y 5.
    \item Preparar y estudiar los conceptos teóricos y prácticos discutidos.
\end{itemize}

\section{Interacción y Participación}
Se anima a los estudiantes a participar activamente en las clases virtuales, utilizando micrófonos y compartiendo pantallas para resolver dudas en tiempo real.

\section{Próxima Clase}
La próxima clase comenzaremos con el tema 6, relacionado con las estructuras de almacenamiento en bases de datos.

\end{document}

