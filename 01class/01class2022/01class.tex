\documentclass{article}
\usepackage[utf8]{inputenc}
\usepackage{amsmath}
\usepackage{hyperref}

\title{Resumen de Clase}
\author{Carlos}
\date{\today}

\begin{document}

\maketitle

\section{Introducción}
En esta clase inicial de Bases de Datos Avanzadas, se discutieron los temas principales del curso, el método de evaluación, y las herramientas necesarias. Además, se revisaron las tareas fundamentales del administrador de bases de datos (DBA) y se presentó el primer ejercicio práctico.

\section{Estructura del Curso}
\begin{itemize}
    \item \textbf{Contacto}: Se proporcionaron los datos del profesor y los medios de comunicación, incluyendo correo electrónico y grupos en Google.
    \item \textbf{Evaluación}:
    \begin{itemize}
        \item Exámenes parciales: 3 o 4 dependiendo del avance.
        \item Ejercicios prácticos: Aproximadamente 20-25 durante el semestre.
        \item Proyecto final: Desarrollo a mitad del semestre en equipos de dos.
        \item Participaciones y puntos extras: Máximo de 30 puntos.
    \end{itemize}
    \item \textbf{Exención de examen final}:
    \begin{itemize}
        \item Promedio mínimo de 7 en exámenes parciales.
        \item Entrega del proyecto final.
        \item Entrega del 80% de los ejercicios prácticos.
    \end{itemize}
\end{itemize}

\section{Primer Ejercicio Práctico}
\begin{itemize}
    \item \textbf{Objetivo}: Instalar el sistema operativo Oracle Linux (versión 8.5 o superior).
    \item \textbf{Requisitos}:
    \begin{itemize}
        \item 50 GB de espacio en disco.
        \item 2-3 GB de RAM.
    \end{itemize}
    \item \textbf{Método}:
    \begin{itemize}
        \item Preferiblemente instalación nativa (no virtualizada).
        \item Manual de instalación disponible en la carpeta compartida de Google Drive.
    \end{itemize}
    \item \textbf{Fecha de entrega}: Próximo viernes.
\end{itemize}

\section{Tareas del Administrador de Bases de Datos}
\begin{itemize}
    \item Evaluar el hardware necesario.
    \item Instalar y configurar el software.
    \item Planear la estructura de la base de datos.
    \item Diseñar estrategias de respaldo.
    \item Administrar usuarios y permisos.
    \item Replicar y clonar bases de datos.
    \item Afinar y mejorar el desempeño de la base de datos.
\end{itemize}

\section{Herramientas de Administración}
\begin{itemize}
    \item \textbf{Línea de Comandos}:
    \begin{itemize}
        \item SQL*Plus
        \item Scripts SQL y Shell
        \item RMAN (Recovery Manager)
    \end{itemize}
    \item \textbf{Herramientas Gráficas}:
    \begin{itemize}
        \item SQL Developer
        \item Oracle Enterprise Manager Express
        \item Herramientas de administración en Cloud (AWS, Azure, Oracle Cloud)
    \end{itemize}
\end{itemize}

\section{Programación en PL/SQL y Shell}
\begin{itemize}
    \item Conocimientos sólidos en SQL y PL/SQL.
    \item Programación en Shell para automatizar tareas administrativas.
    \item Uso de comandos y scripts en entornos Unix/Linux.
\end{itemize}

\section{Siguientes Pasos}
\begin{itemize}
    \item Instalar Oracle Linux siguiendo el manual proporcionado.
    \item Familiarizarse con las herramientas de administración y la programación en PL/SQL y Shell.
    \item Prepararse para el segundo ejercicio práctico, que se realizará en la siguiente clase.
\end{itemize}

\end{document}

